\documentclass[12pt]{report}
\usepackage[utf8]{inputenc}
\usepackage[russian]{babel}
%\usepackage[14pt]{extsizes}
\usepackage{listings}

% Для листинга кода:
\lstset{ %
language=go,                 % выбор языка для подсветки 
basicstyle=\small\sffamily, % размер и начертание шрифта для подсветки кода
numbers=left,               % где поставить нумерацию строк (слева\справа)
numberstyle=\tiny,           % размер шрифта для номеров строк
stepnumber=1,                   % размер шага между двумя номерами строк
numbersep=5pt,                % как далеко отстоят номера строк от подсвечиваемого кода
showspaces=false,            % показывать или нет пробелы специальными отступами
showstringspaces=false,      % показывать или нет пробелы в строках
showtabs=false,             % показывать или нет табуляцию в строках            
tabsize=2,                 % размер табуляции по умолчанию равен 2 пробелам
captionpos=t,              % позиция заголовка вверху [t] или внизу [b] 
breaklines=true,           % автоматически переносить строки (да\нет)
breakatwhitespace=false, % переносить строки только если есть пробел
escapeinside={\#*}{*)}   % если нужно добавить комментарии в коде
}

% Для измененных титулов глав:
\usepackage{titlesec, blindtext, color} % подключаем нужные пакеты
\definecolor{gray75}{gray}{0.75} % определяем цвет
\newcommand{\hsp}{\hspace{20pt}} % длина линии в 20pt
% titleformat определяет стиль
\titleformat{\chapter}[hang]{\Huge\bfseries}{\thechapter\hsp\textcolor{gray75}{|}\hsp}{0pt}{\Huge\bfseries}

%отступы по краям
\usepackage{geometry}
\geometry{verbose, a4paper,tmargin=2cm, bmargin=2cm, rmargin=1.5cm, lmargin = 3cm}
% межстрочный интервал
\usepackage{setspace}
\onehalfspacing
\usepackage{float}
% plot
\usepackage{pgfplots}
\usepackage{filecontents}
\usepackage{amsmath}
\usepackage{tikz,pgfplots}
\usetikzlibrary{datavisualization}
\usetikzlibrary{datavisualization.formats.functions}

\usepackage{graphicx}
\graphicspath{{src/}}
\DeclareGraphicsExtensions{.pdf,.png,.jpg}

\usepackage{geometry}
\geometry{verbose, a4paper,tmargin=2cm, bmargin=2cm, rmargin=1.5cm, lmargin = 3cm}
\usepackage{indentfirst}
\setlength{\parindent}{1.4cm}

\usepackage{titlesec}
\titlespacing{\chapter}{0pt}{12pt plus 4pt minus 2pt}{0pt}
\usepackage{indentfirst}
\setlength{\parindent}{1.4cm}

\usepackage{amsmath}

\begin{document}
%\def\chaptername{} % убирает "Глава"
\tableofcontents
\setcounter{page}{2}

\newpage
\chapter*{Введение}
\addcontentsline{toc}{chapter}{Введение}

Задачи данной лабораторной работы:
\begin{itemize}
	\item Изучить метод Пикара;
	\item реализовать метод Пикарда, явный и неявный методы Эйлера для решения уровнения $y'(x) = x^2 + y^2$;
	\item сравнить методы между собой.
\end{itemize}
\chapter{Аналитическая часть}
В данной части будут рассмотрены теоретические основы методов Пикара, явного и неявного методов Эйлера. 

\section{Постановка задачи} 
Пусть поставлена задача Коши:
\begin{equation*}
	\begin{cases}
	u'(x) = f(x, u(x))\\
	u(x_0) = u_0 \\
	x_0 \leq x \leq x_l
	\end{cases}
\end{equation*}
\section{Метод Пикара}
Данный метод является представителем класса приближенных методов решения.
Идея метода чрезвычайно проста и сводится к процедуре последовательных приближений для решения интегрального уравнения, к которому приводится исходное дифференциальное уравнение.

Проинтегрируем выписанное уравнение

\begin{equation}
	u(x) = u_0 + \int_{x_0}^{x} f(t,u(t))dt
\end{equation}

Процедура последовательных приближений метода Пикара реализуется согласно следующей схеме.
\begin{equation}
y_s(x) = u_0 + \int_{x_0}^{x} f(t,y_{s-1}(t))dt
\end{equation}
причем $y_0(t) = u_0$

\section{Явный метод Эйлера}
Самый простой метод решения уровнения - дискретизация расчетного интервала и замена производной в левой части $\frac{du(x)}{dt}$	разностным аналогом. Для некоторой i-ой точки сетки разностная производная определяется следующим образом:
\begin{equation}
\frac{du(t_i)}{dt} \approx \frac{y_{i+1} - y_i}{h}
\end{equation}
Для того, чтобы схема имела простое решение, правую часть уравнения возьмем в той же точке $t_i$:
\begin{equation}
\frac{y_{i+1} - y_i}{h} = f(t_i, y_i)
\end{equation}
Таким образом, мы сразу получаем рекуррентную формулу определения нового значения y в точке $t_{i+1}$, т.е. $y_{i+1}$ по значению y в точке $t_{i+1}$. Это значение обозначим как $y_{i}$, а $y_{i+1}$ запишем как:
\begin{equation}
y_{i+1} = y_i + h \cdot f(x_i,y_i)
\end{equation}
\section{Неявный метод Эйлера}
\begin{equation}
y_{i+1} = y_i + h \cdot f(x_i+1,y_i+1)
\end{equation}
Геометрическая интерпретация  одного шага этого метода заключается в том, что решение на отрезке $[t_i;t_{i+1}]$ аппроксимируется касательной $y = y_{i+1} + y'(t_{i+1})(t-t_{i+1})$, проведенной в точке $(t_{i+1},y_{i+1})$ к интегральной кривой, проходящей через эту точку.
\chapter{Технологическая часть}

\section{Выбор ЯП}
В качестве языка программирования был выбран golang.
Время работы алгоритмов было замерено с помощью time. 
\section{Листинг кода алгоритмов}
В данном разделе будут приведены листинги кода решения методом Пикара (Листинг \ref{picard}) и Эйлера (Листинг \ref{euler})
\begin{lstlisting}[label=picard,caption = Метод Пикара, language = go]

func picard(x float64, n int)float64{
	u0 := 0.0
	answer := 0.0
	poly := make(map[int]float64)
	curr := make(map[int]float64)
	poly[2] = 1.0
	var res float64
	for i:=0;i<n;i++{
		curr = poly_pow(curr)
		curr[2] = 1.0
		curr, res = integrate(curr, 0.0, x)
		answer = u0 + res
	}
	return answer
}
\end{lstlisting}

\begin{lstlisting}[label=euler,caption = Явный и неявный методы Эйлера, language = go]
func euler_explicit(xn float64, n int)float64{
	h := xn / float64(n)
	y:=0.0 
	x:=0.0
	for i:=0; i<=n;i++{
		y = y + h*f(x, y)
		x+=h
	}
	return y
}
func euler_implicit(xn float64, n int)float64{
	h := xn / float64(n)
	y:=0.0 
	x:=0.0
	var a, b, c, dis, x1 float64
	for i:=0;i<=n;i++{
		a = 1; b = -1.0/h; c = 1.0/h*y+(x+h)*(x+h)
		dis = D(a, b, c)
		if dis>=0{
			x1 = (-b - math.Sqrt(dis))/2/a
		}
		y = x1
		x+=h
	}
	return y
}
\end{lstlisting}


\end{document}

